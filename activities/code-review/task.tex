\documentclass[12pt,a4paper]{article}
\usepackage[english]{babel}
\usepackage[utf8]{inputenc}
\usepackage{fullpage}
\usepackage{cmap}		% Make PDF file search­able and copy­able (ASCII characters)
\usepackage{lmodern}	% Make PDF file search­able and copy­able (Accented characters)
\usepackage[T1]{fontenc}	% Hyphenate accented words
\usepackage{inconsolata}
\usepackage{amsmath,amsfonts,amssymb} % Mathematics
\usepackage[protrusion]{microtype}	% Better typeset results
\usepackage[autostyle,english=american]{csquotes}
\usepackage[table]{xcolor}
\usepackage{enumitem}
\usepackage{pbox}

\pagenumbering{gobble}

% Typesetting code
\definecolor{base-blue}{RGB}{13, 91, 201}
\definecolor{keywords}{RGB}{129, 19, 8} % red
\definecolor{identifiers}{RGB}{19, 8, 129} % blue
\definecolor{strings}{RGB}{8, 129, 19} % green
\definecolor{comments}{RGB}{104, 104, 110} % gray
\definecolor{error}{RGB}{220, 50, 47} % bright red
\usepackage{listings}	% listlisting environment
\usepackage{upquote}
\lstset{
  basicstyle=\ttfamily\small\bfseries,
  breakatwhitespace=false,
  breaklines=true,
  captionpos=b,
  commentstyle=\ttfamily\color{comments}\bfseries,
  escapeinside={\$}{)},
  identifierstyle=\color{identifiers},
  keywordstyle=\ttfamily\color{keywords},
  language=Python,
  mathescape=true
  numbers=none,
  showspaces=false,
  showstringspaces=false,
  showtabs=false,
  stringstyle=\ttfamily\color{strings},
  tabsize=2,
  aboveskip=-0.3em,
  belowskip=-0.3em,
  xleftmargin=2.15em,
}
% Frames around code lstlistings
\usepackage[framemethod=TikZ]{mdframed}
\usepackage{etoolbox}
\mdfdefinestyle{listingstyle}{%
	innerleftmargin=0pt,
	innerrightmargin=0pt,
	rightmargin=0pt,
	linewidth=0pt,
	backgroundcolor=base-blue!5!white,
}
\BeforeBeginEnvironment{lstlisting}{\begin{mdframed}[style=listingstyle]}
\AfterEndEnvironment{lstlisting}{\end{mdframed}}

\def\arraystretch{1.5}

\title{\textbf{Introductory Python Programming}}
\author{}
\date{}

\begin{document}

\setlist[itemize]{itemsep=0ex}

\vspace*{-1.35cm}
\noindent
Program a simulator of a simple game of the type ``Mensch ärgere Dich nicht'':

\begin{verbatim}
https://en.wikipedia.org/wiki/Mensch_%C3%A4rgere_Dich_nicht
\end{verbatim}

\begin{itemize}
\item Play with one piece on a game board with size $n\in\{1, 2, 3, \dots\}$ squares.
\item The starting position is the 0th square (outside the board), the ending position is the $n$th square.
\item You roll a single dice with values from 1 to 6.
\begin{itemize}
\item The piece moves by the rolled number of squares (not outside the board).
\item If 6 is rolled, the roll repeats (within the same round).
\item If 6 is rolled, but the movement of the piece would end outside the board, the round ends.
\item If 6 and then X is rolled, but the movement of 6 + X would end outside the board, the round ends (even if movement by 6 would be possible).
\item If three 6s are rolled in a row, the round ends and the piece moves to the starting position.
\end{itemize}
\item The game ends when the piece exactly reaches the ending position.
\end{itemize}
The main function is \texttt{game(board\_size, print\_state=True, probability\_of\_six=1/6)}.

\begin{itemize}
\item \texttt{board\_size} corresponds to the $n$ mentioned above.
\item \texttt{print\_state} controls verbosity. If it is \texttt{True}, the function prints the text on the screen and does not return anything. If it is \texttt{False}, the function does not print anything and returns the length of the game (the number of rounds until the finish).
\item \texttt{probability\_of\_six} sets the chance of rolling 6 (other probabilities are uniform).
\end{itemize}
Example game run and output:

\begin{lstlisting}[deletekeywords={in,round}]
>>> game(20)
Round 1 -- Roll: 6 5 
I am on position 11
Round 2 -- Roll: 6 5 
I am on position 11
Round 3 -- Roll: 5 
I am on position 16
Round 4 -- Roll: 5 
I am on position 16
Round 5 -- Roll: 2 
I am on position 18
Round 6 -- Roll: 4 
I am on position 18
Round 7 -- Roll: 2 
I am on position 20
Game finished in Round 7.
>>> game(20, False)
11
\end{lstlisting}

\end{document}